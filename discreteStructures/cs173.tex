\documentclass[a4paper]{article}
\usepackage{amsmath,amssymb,enumerate,gastex, makeidx, graphicx,amsfonts, hyperref}
\begin{document}

\begin{center}
{\bf{\huge CS 173 }}\\
{\bf{\huge a.k.a. Discrete Structures}}\\
\vspace{ 5 cm }
{\bf{\large Spring 2011, University of Illinos}}\\
{\bf{\large as taught by Margaret Fleck}}\\
\vspace{ 1 cm }                         
{\large notes by Michele R. Esposito}\\
\vspace{ 5cm }
Please, let report any error or type-o at \underline{\href{mailto:micheleresposito@gmail.com}{ micheleresposito@gmail.com }}

\end{center}

\newpage
\tableofcontents
\newpage
% =========================================================
\section{Logic}
{\bf Converse:} $p \rightarrow q$ becomes $q \rightarrow p$. It is not logically equivalent.\\
{\bf Contrapositive:} $p \rightarrow q$ becomes $\neg q \rightarrow \neg p$. Logically equivalent.\\
{\bf Negation implies:} $\neg (p \rightarrow q) \equiv \neg p \vee q$\\
{\bf De Morgan's Law} $\neg (p \wedge q) \equiv \neg p \vee \neg q$.\\
$\neg (p \vee q) \equiv \neg p \wedge \neg q$.\\

% =========================================================
\section{Sets}
{\bf Definition:} A set is an unordered collection of objects.\\
{\bf Cardinality:} Is how many objects there are in a set: $A=\{1,3,4\}$ then $|A|=3$.\\
{\bf Cartesian Product:} A x B= $\{(x,y)| x\in A$ and $y\in B\}$.\\
{\bf Size of set union:} $|A \cup B|= |A|+|B|$.\\
{\bf Product rule:} if $|A|=n, |B|=q$ then $|A $x $B|=nq$.\\
% =========================================================

\section{Number theory}
{\bf Divisibility} $a$ divides $b$ if $b=an$ for some integer $n$. The short hand for $a$ divedes b is $ a|b$.\\
{\bf Prime numbers} an integer $q \geq 2$ is prime if the only positive factors of $q$ are $q$ and 1. An integer $q\geq 2$ is composite if it is not prime.\\

{\bf Fundamental Theorem of Arithmetics:} Every integer $\geq 2$ can be written as the product of one or more prime factors. Except for the order in which you write the factors, this prime factorization is unique.\\
{\bf GCD:} Greatest Common Divisor. The shorthand is gcd(a,b). It is done my taking the prime factors and xracting hte shared factors.\\
{\bf LCM:} Least Common Multiple. $lcm(a,b)=\frac{ab}{gcd(a,b)}$. If two integers share no factors, $gcd(a,b)=1$ then they are called {\bf relatively prime}.\\

{\bf Congruence mod k:} if $k$ is any positive integer, two integers $a$ and $b$ are congruent mod $k$ (written $a \equiv b (mod k)$) if $k|(a-b)$, thus $a-b=kn$.

% =========================================================
\section{Relations}
A relation R on a set A is a subset of $A^2$, thus R is a set of ordered pairs of elements from A.\\
{\bf Reflexive:} Every element is related to itself. $\forall x\in A, xRx $.\\
{\bf Irreflexive:} No element is related to itself. $\forall x\in A, x\not R x$.\\
{\bf Symmetric:} $\forall x,y \in A, xRy \rightarrow yRx$.\\
{\bf Antisymmetric:} $\forall x,y \in A$ with $x \neq y, if xRy,$ then $y \not R x$.\\
$\forall x,y \in A, if xRy \wedge yRx,$ then $ x=y$.\\
{\bf Transitive:} $\forall a,b,c \in A, aRb \wedge bRc \rightarrow aRc$.\\
{\bf Partial order:} A relation that is transitive, reflexive and antisymmetric.\\
{\bf Linear (total) order:} is a partial order in which every pair of elements are comparable. That is, $\forall x,y, x\geq y \vee y\geq x$.\\
{\bf Strict partial order:} is a relation that is irreflexive, antisymmetric and transitive.\\
{\bf Equivalence relation:} is a relation that is reflexive, symmetric and transitive.\\
{\bf Equivalence class:} Is the set of all elements related to x. $[x]_R=\{y \in A | xRy\}$.\\
% =========================================================
\section{ Functions}
{\large {\bf Onto}}\\
{\bf Definition:} A function $f : A \rightarrow B$ is $onto$ or $surjective$ if its image is its whole co-domain. If every element in the co-domain if mapped in the domain. Or, equivalently,
\begin{center}
$\forall y\in B, \exists x \in A, f(x)=y$
\end{center}
Say you want to compose the function $f : A \rightarrow B$ and $g : B \rightarrow C$. Then $g o f: A \rightarrow C$.

{\large {\bf One-To-One}}\\
A function is $one-to-one$ or $injective$ if it never assigns two input values to the same output value. Its formal definition is:
\begin{center}
$\forall x,y \in A, x\neq y \rightarrow f(x) \neq f(y)$\\
\end{center}
or, by taking the contrapositive,
\begin{center}
$forall x,y\in A, f(x) = f(y) \rightarrow x=y$\\ 
\end{center}

% =========================================================
\section{Graphs}
Graphs are a very general class of objects. A graph consists of a set of nodes V and a set of edges E. Two nodes connected by an edge are called {\bf neighbors} or {\bf adjacent}.\\
{\bf Undirected graph:} when edges can be crossed in both directions.\\
{\bf Simple graph:} If it has neither multiple edges nor loop edges.\\
{\bf Degrees:} the degree of a vertex $v$, written $deg(v)$ is the number of edges which have $v$ as an endpoint.\\
{\bf Handshaking Theorem:} $\sum_{v \in V} deg(v)= 2|E|$.\\
{\bf Complete graph:}$ K_n$ is a graph in which every vertex is connected to another vertex.\\
{\bf Cycle:} $C_n$. It is a graph that is only a cycle.\\
{\bf Wheels:} $W_n$ is just like $C_n$ except that it has an additional central "hub" node which is connected to all the others.\\
{\bf Isomorphism:} An isomorphism from $G_1$ to $G_2$ is a bijection $f: V_1 \rightarrow V_2$ such that vertices $a$ and $b$ are joined by an edge if and only if $f(a)$ and $f(b)$ are joined by an edge.\\
{\bf Walk:} a walk of legth $k$ from bertex $a$ to vertex $b$ is a sequence of vertices and a sequence of edge that connects them.\\
 {\bf Closed walk:} if the starting and ending verticies are the same. Otherwise is {\bf open}.\\
{\bf Path:} A path is an open walk in which no vertex is used more than one.\\
{\bf Cycle:} Is a closed walk with at least three vertices in which no vertex is used more than once except that the starting and ending vertices are the same.
{\bf Connected Graph:} A graph is connected if there is a walk between every pair of nodes.\\
{\bf Cut edge:} An edge that if its cut will increase the number of connected componets.\\
{\bf Diameter:} Is the maximum distance between any pair of nodes in hte graph.\\
{\bf Euler circuits:} When there is a closed walk that uses each edge of the graph exactly once. It is possible whenever the degree of every node in the graph is ever.\\
{\bf Bipartite graph:} Is a graph that can be colored with two colors.
% =========================================================
\section{Induction}
{\bf Hypercube:} A $n$-cube or a hypercube $Q_n$ os the graph of the of the corners and edges of an $n$-dimentional cube.\\
% =========================================================
\section{Trees}
{\bf Definition:} Finite, non-empty set of nodes plus a set of edges. each edge connects each node to its {\bf parent} node.\\
{\bf Root:} A node with no parent.\\
{\bf Child:} if x is the parent of y, then y is called a child of x. It z is also a child of x, then y and x are {\bf siblings}.\\
{\bf Leaf node} is a node that has no children.\\
{\bf Internal node:} a node that does have a children.\\
{\bf Levels:} The nodes of a tree can be organized into levels, based on how many edges away from the root htey are.\\
{\bf Height:} maximum level of any of its node, or maximum length of a path from the root to a leaf.\\
{\bf m-ary trees:} A m-ary tree is the restriction on how many child a node can have.\\
{\bf Heap property:} if, for every node X in the tree, the value of X is at least as big as the value in each of X's children.

% =========================================================
\section{ Big-O}
{\bf Definition:} f(x) is O(g(x)) if and only if there are positive real numbers $c,k$ such that $0\leq f(x) \leq cg(x), \forall x\geq k$.\\
{\bf Omega f(x):} $g(x)$ is $\Omega (f(x))$ if and only if $f(x)$ is $O(g(x))$. Which means that if $g(x) \succeq f(x)$.\\
{\bf Theta g(x):} f(x) is $\Theta (g(x))$ if and only if $g(x)$ is $O(f(x))$ and $f(x)$ is $O(g(x))$. Therefore is the equivalence relation.
% =========================================================
\section{Algorithms}

{\large {\bf Linear search} }\\
Go through the array, find the minimum value.\\
{\bf Average Case:} $O(n)$.\\
{\bf Best Case:} $O(n)$.\\

{\large {\bf Binary search}}\\
Given a sorted array, returns the index of the given value.\\
{\bf Running time:} $O(\log n)$\\.

{\large {\bf Insertion sort}}\\
Devide the input array into two pieces and orders it.\\
{\bf Running time:} $O(n^2)$.\\

{\large {\bf Mergesort}}\\
Mergesort devides the big input list into smaller pieces, and hten it merges them back together. It is a very stable and very predictable algorithm.\\
{\bf Running time:} $O(n \log_n)$.\\

{\large {\bf Katatsuba's algorithm}}\\
First fast algorithm for multiplying big numbers together. It works by breaking up numbers and them multiplieng them together, using the recurrence $T(n)= 3T(\frac{n}{2})+O(n)$, which gives you $O(n^{\log_2 3})$.

% =========================================================
\section{Sets of Sets}
{\bf Powerset:} Is th set containing all subsets of A.\\
{\bf Partition:} when we divide a base set A into non-overlapping subsets, the result is called a $partition$.\\
{\bf Binomial Theorem:} Let $x$ and $y$ be varibales and let $n$ be any natural number. Then\\
\begin{center}
$(x+y)^n=\sum^n_{k=0} {n \choose k}  x^{n-k} y^k$
\end{center}
% =========================================================
\section{Countability}
{\bf Cardinality:} Two sets A and B have the same cartinality ($|A|=|B|$) if an only if there is a bijection from A to B.\\
{\bf Countable:} An infinite set A is $countable$ or $countably infinite$ if there is a bijection from $\mathbb{N}$ onto A.\\
{\bf Cantor Schroeder Bernstein Theorem:} $|A| \leq |B|$ if and only if there is a one-to-one function from A to B. Then if $|A| \leq |B|$ and $|B| \leq |A|$, $|A|=|B|$.

 % =========================================================
\section{Planar Graphs}
{\bf Definition:} A planar graph is a graph which can be drawn in hte plane without any edges crossing.\\
{\bf Degree of a face} is the length of its boundary, how many edges it has.\\
{\bf Euler's formula:} $v-e+f=2$.\\
{\bf Handshaking theorem for faces:} sum of the faces degrees is also 2$e$.\\
{\bf Free tree:} Any connected graph with no cycles.\\
{\bf Corollay of Euler's formula:} In a graph where $v \geq 3, e \leq 3v-6$.\\
{\bf Subdivision:} In a graph G a $subdivision$ of another graph F is G just like F except that you've divided up some of F's edges by adding vertces in the middle of them.\\
{\bf Homeomorphic:} Two graphs are $homeomorphic$ if one is a subdivision of another, or they are both subdivision of some third graph.\\
{\bf Kuratowski's theorem:} A graph is nonplanar if and only if it contains a subgraph homeomorphic to $K_{3,3}$ or $K_5$.\\
 % =========================================================

\end{document}
